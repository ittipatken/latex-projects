\documentclass[a4paper, 12pt]{article}

\usepackage{fontspec}
\setmainfont{STIX Two Text}
\newfontfamily{\defaultfont}{STIX Two Text}
\newfontfamily{\thaifont}[Scale=MatchUppercase]{TH Sarabun Chula}

\usepackage{polyglossia}
\setdefaultlanguage{thai}
\setotherlanguages{english}

\usepackage[Latin, Thai]{ucharclasses}
\setDefaultTransitions{\defaultfont}{}
\setTransitionTo{Thai}{\thaifont}

\XeTeXlinebreaklocale "th_TH"
\XeTeXlinebreakskip = 0pt plus 1pt

\linespread{1.25}

\usepackage{amsmath, amsthm, amssymb}
\usepackage{unicode-math}
\setmathfont{STIX Two Math}
\usepackage[margin=2cm]{geometry}

\DeclareMathOperator{\Var}{Var}

\begin{document}
\title{Standard Error Derivation}
\author{อิธิพัฒน์ ธนบดีกาญจน์}

\maketitle

\begin{abstract}
	Standard Error (SE) หรือ Standard Error of Sample Mean (SEM) หมายถึง ค่าที่แสดงว่าโดยเฉลี่ยแล้วค่าเฉลี่ยของตัวอย่างแต่ละตัวแตกต่างจากค่าเฉลี่ยของประชากรมากน้อยเพียงใด หรือเขียนในรูปแบบสมการคณิตศาสตร์ได้ว่า \(\text{Standard Error} = \sqrt{\Var(\mu)}\) โดย \(\mu = \text{mean of sample mean}\) และสามารถคำนวณได้จาก \(s/\sqrt{n}\) ซึ่งสามารถพิสูจน์ได้โดยใช้สมบัติของ variance
\end{abstract}
\section{พิสูจน์}
กำหนด sample mean หรือค่าเฉลี่ยของค่าเฉลี่ย เป็น \(\bar{x}_1, \bar{x}_2, \bar{x}_3,\dots, \bar{x}_k\)\\
จากนิยาม
\begin{equation}
	\text{Standard Error} = \sqrt{\Var(\mu)} = \sqrt{\Var\left(\frac{\sum_{i=1}^{k}\bar{x}_k}{k}\right)}
\end{equation}
ใช้สมบัติของ variance เมื่อ แต่ละ sample ไม่ขึ้นต่อกัน (independent) และ \(\Var(aX) = a^2\Var(X)\) จึงได้ว่า
\begin{equation*}
	\Var\left(\frac{\sum_{i=1}^{k}\bar{x}_k}{k}\right)=\frac{1}{k^2}\sum_{i=1}^{k}\Var(\bar{x}_k)=\frac{1}{k^2}\sum_{i=1}^{k}s^2
\end{equation*}
เนื่องจาก \(s^2\) ไม่ขึ้นกับ \(k\) ดังนั้น
\begin{equation*}
	\frac{1}{k^2}\sum_{i=1}^{k}s^2=\frac{1}{k^2}(ks^2)=\frac{s^2}{k}
\end{equation*}
\begin{equation}
	\text{Standard Error} = \sqrt{\frac{s^2}{k}} = \frac{s}{\sqrt{k}}\;\square
\end{equation}
\end{document}