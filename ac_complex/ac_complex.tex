\documentclass[a4paper,12pt]{article}

\usepackage{fontspec}
\newfontfamily{\defaultfont}{CMU Serif}
\newfontfamily{\thaifont}[Scale=MatchLowercase]{TH Sarabun Chula}

\usepackage{polyglossia}
\setdefaultlanguage{thai}
\setotherlanguages{english}

\usepackage[Latin,Thai]{ucharclasses}
\setDefaultTransitions{\defaultfont}{}
\setTransitionTo{Thai}{\thaifont}

\XeTeXlinebreaklocale "th"
\XeTeXlinebreakskip = 0pt plus 0pt

\linespread{1.25}

\usepackage{amsmath,amsthm,amssymb}
\usepackage{physics}
\usepackage{circuitikz}
\usepackage[margin=1in]{geometry}
\usepackage{graphicx}

\renewcommand{\thefootnote}{\fnsymbol{footnote}}

\title{การแก้ปัญหาวงจรไฟฟ้ากระแสสลับด้วยจำนวนเชิงซ้อน}
\author{อิธิพัฒน์ ธนบดีกาญจน์}

\begin{document}
\maketitle
\section{บทนำ}
การใช้จำนวนเชิงซ้อนดีกว่าการใช้ Phasor ในระบบมัธยมปลายทั่วไปอย่างไร ?
\begin{itemize}
	\item ไม่ต้องวาด Phasor Diagram
	\item เป็นการแก้ปัญหาด้วยพีชคณิต ไม่ต้องใช้เรขาคณิตซึ่งโดยส่วนใหญ่แล้วพีชคณิตจะง่ายกว่า
\end{itemize}

วิธีการคิด จะใช้ Euler's formula\footnote{ในระดับมัธยมปลายจะนิยมเขียน $\mathrm{cis}\,\theta$} ช่วยในการแปลงตรีโกณมิติเป็นจำนวนเชิงซ้อน
	\begin{equation}
		e^{j\theta}=\cos\theta+j\sin\theta
	\end{equation}
เราจะใช้ $j$ เพื่อสงวน $i$ ไว้ใช้เป็นตัวแปรของกระแสไฟฟ้า
	\begin{equation}
		j^2=-1
	\end{equation}
แต่เนื่องด้วย วิธีการคิดที่เป็นจำนวนเชิงซ้อนแต่คำตอบที่ต้องการเป็นจำนวนจริง เราจึงต้องพิจารณาแหล่งจ่ายไฟฟ้ากระแสสลับว่า เป็นฟังก์ชันของ $\sin$ หรือ $\cos$
และนำเฉพาะส่วนจริง (Real Part) หรือส่วนจินตภาพ (Imaginary Part) เป็นคำตอบ
\section{การใช้ปริมาณเชิงซ้อนในไฟฟ้ากระแสสลับ}
 \begin{enumerate}
	\item แหล่งจ่ายไฟฟ้ากระแสสลับ
	\begin{enumerate}
		\item $V_{\mathrm{max}}\sin\omega t\rightarrow V_{\mathrm{max}}e^{j\omega t}$ โดย $V_{\mathrm{max}}\sin\omega t=\Im\left[V_{\mathrm{max}}e^{j\omega t}\right]$
		\item $V_{\mathrm{max}}\cos\omega t\rightarrow V_{\mathrm{max}}e^{j\omega t}$ โดย $V_{\mathrm{max}}\cos\omega t=\Re\left[V_{\mathrm{max}}e^{j\omega t}\right]$
	\end{enumerate}
	\item ตัวต้านทาน
	\begin{enumerate}
		\item $R\rightarrow R$
	\end{enumerate}
\newpage
	\item ตัวเก็บประจุ
	\begin{enumerate}
		\item $C\rightarrow \tilde{X}_{C}=\dfrac{1}{j\omega C}$
	\end{enumerate}
		\item ตัวเหนี่ยวนำ
	\begin{enumerate}
		\item $L\rightarrow \tilde{X}_{L}=j\omega L$
	\end{enumerate}
\end{enumerate}
\section{ตัวอย่างการแก้โจทย์ปัญหา}
	\subsection{วงจร R L C ต่ออนุกรม}
	\begin{center}
		\begin{circuitikz}
			\draw
			(0,0) to[sinusoidal voltage source] (6,0) -- (6,2)
			to[C=$C$] (4,2)
			to[L=$L$] (2,2)
			to[R=$R$] (0,2) -- (0,0); 
		\end{circuitikz}
	\end{center}
ให้ $v(t)=V_{\mathrm{max}}\sin\omega t$ คร่อมแหล่งจ่ายไฟฟ้ากระแสสลับ จงหา $i(t)$\\
\textbf{วิธีทำ} ให้ 
\begin{align*}
\tilde{v}(t)&=V_{\mathrm{max}}e^{j\omega t}\\
v(t)&=\Im\left[V_{\mathrm{max}}e^{j\omega t}\right]
\end{align*}
กำหนดให้ $Z$ คือ ความขัด (Impedance) ของ $R,\,X_{L}$ และ $X_{C}$\\
\begin{align*}
\tilde{Z}&=R+\tilde{X}_{L}+\tilde{X}_{C}\\
\tilde{Z}&=R+j\omega L+\frac{1}{j\omega C}\\
\tilde{Z}&=R+j\omega L-\frac{j}{\omega C}\\
\tilde{Z}&=R+\left(\omega L-\frac{1}{\omega C}\right)j
\end{align*}
แปลงให้อยู่ใน รูปเชิงขั้ว
\begin{equation*}
\tilde{Z}=\sqrt{R^2+\left(\omega L-\frac{1}{\omega C}\right)^2}e^{j\arctan(\frac{\omega L-1/\omega C}{R})}
\end{equation*}
จาก $\tilde{v}=\tilde{i}\tilde{Z}$ (Ohm's law)\\
\begin{align*}
\tilde{i}(t)&=\dfrac{\tilde{v}(t)}{\tilde{Z}}\\
\tilde{i}(t)&=\dfrac{V_{\mathrm{max}}e^{j\omega t}}{\sqrt{R^2+\left(\omega L-\frac{1}{\omega C}\right)^2}e^{j\arctan(\frac{\omega L-1/\omega C}{R})}}\\
\tilde{i}(t)&=\dfrac{V_{\mathrm{max}}}{\sqrt{R^2+\left(\omega L-\frac{1}{\omega C}\right)^2}}e^{j\left(\omega t-\arctan(\frac{\omega L-1/\omega C}{R})\right)}
\end{align*}
take imaginary part
$$\therefore i(t)=\dfrac{V_{\mathrm{max}}}{\sqrt{R^2+\left(\omega L-\frac{1}{\omega C}\right)^2}}\sin(\omega t-\arctan(\frac{\omega L-1/\omega C}{R}))\;\blacksquare$$
	\subsection{วงจร R L C ต่อขนาน}
	\begin{center}
		\begin{circuitikz}
			\draw
			(0,0) to[sinusoidal voltage source] (0,2.5)
			to[short] (2,2.5)
			to[R=$R$] (2,0)
			to[short] (0,0);
			\draw (2,2.5)
			to[short] (4,2.5)
			to[L=$L$] (4,0)
			to[short] (2,0);
			\draw (4,2.5)
			to[short] (6,2.5)
			to[C=$C$] (6,0)
			to[short] (4,0); 
		\end{circuitikz}
	\end{center}
ให้ $v(t)=V_{\mathrm{max}}\cos\omega t$ คร่อมแหล่งจ่ายไฟฟ้ากระแสสลับ จงหา $i(t)$\\
\textbf{วิธีทำ} ให้
\begin{align*}
\tilde{v}(t)&=V_{\mathrm{max}}e^{j\omega t}\\
v(t)&=\Re\left[V_{\mathrm{max}}e^{j\omega t}\right]
\end{align*}
กำหนดให้ $Z$ คือ ความขัด (Impedance) ของ $R,\,X_{L}$ และ $X_{C}$
\begin{align*}
\dfrac{1}{\tilde{Z}}&=\dfrac{1}{R}+\dfrac{1}{\tilde{X}_C}+\dfrac{1}{\tilde{X}_L}\\
\dfrac{1}{\tilde{Z}}&=\dfrac{1}{R}+j\omega C+\dfrac{1}{j\omega L}\\
\dfrac{1}{\tilde{Z}}&=\dfrac{1}{R}+\left(\omega C-\dfrac{1}{\omega L}\right)j
\end{align*}
แปลงให้อยู่ใน รูปเชิงขั้ว
\begin{equation*}
\dfrac{1}{\tilde{Z}}=\sqrt{\dfrac{1}{R^2}+\left(\omega C-\dfrac{1}{\omega L}\right)^2}e^{j\arctan(\frac{\omega C-1/\omega L}{1/R})}
\end{equation*}
\noindent จาก $\tilde{v}=\tilde{i}\tilde{Z}$ (Ohm's law)\\
\begin{align*}
\tilde{i}(t)&=\dfrac{\tilde{v}(t)}{\tilde{Z}}\\
\tilde{i}(t)&=V_{\mathrm{max}}e^{j\omega t} \sqrt{\dfrac{1}{R^2}+\left(\omega C-\dfrac{1}{\omega L}\right)^2}e^{j\arctan(\frac{\omega C-1/\omega L}{1/R})}\\
\tilde{i}(t)&=V_{\mathrm{max}}\sqrt{\dfrac{1}{R^2}+\left(\omega C-\dfrac{1}{\omega L}\right)^2}e^{j\arctan(\omega t+\frac{\omega C-1/\omega L}{1/R})}
\end{align*}
take real part
$$\therefore i(t)=V_{\mathrm{max}}\sqrt{\dfrac{1}{R^2}+\left(\omega C-\dfrac{1}{\omega L}\right)^2}\cos(\omega t+\arctan(\frac{\omega C-1/\omega L}{1/R}))\;\blacksquare$$\\
\begin{thebibliography}{9}
	\bibitem{physicschula} 
	จุฬาลงกรณ์มหาวิทยาลัย. คณะวิทยาศาสตร์. ภาควิชาฟิสิกส์. (2559). \underline{ฟิสิกส์ 2}. พิมพ์ครั้งที่ 14. กรุงเทพฯ: สำนักพิมพ์จุฬาลงกรณ์มหาวิทยาลัย.
	\bibitem{physicsipst4}
	\raggedright สถาบันส่งเสริมการสอนวิทยาศาสตร์และเทคโนโลยี กระทรวงศึกษาธิการ. (2554). \underline{หนังสือเรียน รายวิชาเพิ่มเติม ฟิสิกส์ เล่ม 4}. พิมพ์ครั้งที่ 1. กรุงเทพฯ: โรงพิมพ์ สกสค. ลาดพร้าว.
	\bibitem{genphys2midterm}
	Nossu G. N. (2562). \underline{GEN Phys 2 Midterm}.
\end{thebibliography}

\end{document}