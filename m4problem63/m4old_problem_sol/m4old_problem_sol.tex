\documentclass[a4paper,12pt]{article}

\usepackage{fontspec}
\newfontfamily{\defaultfont}{CMU Serif}
\newfontfamily{\thaifont}[Scale=MatchLowercase]{TH Sarabun Chula}

\usepackage{polyglossia}
\setdefaultlanguage{thai}
\setotherlanguages{english}

\usepackage[Latin,Thai]{ucharclasses}
\setDefaultTransitions{\defaultfont}{}
\setTransitionTo{Thai}{\thaifont}

\XeTeXlinebreaklocale "th"
\XeTeXlinebreakskip = 0pt plus 0pt

\linespread{1.25}

\usepackage{amsmath,amsthm,amssymb}
\usepackage[ISO]{diffcoeff}
\usepackage{siunitx}
\usepackage[margin=1in]{geometry}
\usepackage{graphicx}
\usepackage{hyperref}

\DeclareMathOperator{\cosec}{cosec}
\pagenumbering{gobble}

\begin{document}
\begin{center}
	\textbf{{\Large เฉลย ข้อ 1}}
\end{center}
\begin{enumerate}
	\item \begin{align*}
		      W_F          & =\;\text{พิื้นที่ใต้กราฟ}   \\
		      W            & =\frac{1}{2}(2L)(2mg) \\
		      \therefore W & =2mgL
	      \end{align*}
	\item \begin{align*}
		      W_f            & =-\mu NL  \\
		      \therefore W_f & =-\mu mgL
	      \end{align*}
	\item \begin{align*}
		      \frac{1}{2}mv^2 & =W_F+W_f           \\
		      \frac{1}{2}mv^2 & =2mgL-\mu mgL      \\
		      v^2             & =4gL-2\mu gL       \\
		      \therefore v    & =\sqrt{2gL(2-\mu)}
	      \end{align*}
\end{enumerate}
\newpage
\begin{center}
	\textbf{{\Large เฉลย ข้อ 2}}\\
\end{center}
พิจารณา มวล \(m\)\\
กำหนด ช่วง 1 คือช่วงก่อนถึงจุด A, ช่วง 2 คือช่วงเริ่มเกี่ยวปลายสปริงและช่วง 3 คือช่วงที่ลอยขึ้นสูงสุด\\
\begin{enumerate}
	\item 
	      \begin{align*}
	      	W_{\text{nc}}&=E_2-E_1\\
		      -\mu mg\ell&=\frac{1}{2}mv^2-\frac{1}{2}mu_0^2 \\
		      -2\mu g\ell&=v^2-6\mu g\ell                    \\
		      v^2&=4\mu g\ell                                \\
		      \therefore v&=\sqrt{4\mu g\ell}                \\
	      \end{align*}
	\item 
	      \begin{align*}
	      	W_{\text{nc}}&=E_3-E_2\\
		      0&=\left( \frac{1}{2}kx^2+mgy\right) -\frac{1}{2}mv^2                                               \\
		      \frac{1}{2}m(4\mu g\ell)
		      &=\frac{1}{2}\left( \frac{\mu mg}{\ell}\right) \left( 2\ell-\ell\right) ^2+mgy \\
		      4\mu \ell&=\frac{\mu}{\ell}\left( \ell\right) ^2+2y                                          \\
		      4\mu \ell&=\mu \ell+2y                                                       \\
		      \therefore y&=\frac{3\mu \ell}{2}
	      \end{align*}
\end{enumerate}
\newpage
\begin{center}
	\textbf{{\Large เฉลย ข้อ 3}}
\end{center}
\begin{enumerate}
	\item \begin{align*}
		      W_{\text{nc}}                       & =E_\text{f}-E_\text{i}       \\
		      F\Delta x                           & =E_\text{f}-E_\text{i}       \\
		      mg\left(\frac{H}{\sin\alpha}\right) & =\frac{1}{2}mv^2+mgH         \\
		      \therefore v                        & =\sqrt{2gH(\cosec \alpha-1)}
	      \end{align*}
	\item \begin{align*}
		      P            & =\frac{W}{\Delta t}            \\
		                   & =\frac{F\Delta x}{\Delta t}    \\
		                   & =Fv                            \\
		      \therefore P & =mg\sqrt{2gH(\cosec \alpha-1)}
	      \end{align*}
	\item \begin{align*}
		      W_{\text{nc}}                                                                           & =E_\text{f}-E_\text{i}                 \\
		      F\left( \frac{H}{\sin\alpha}\right) -\mu mg\cos\alpha\left( \frac{H}{\sin\alpha}\right) & =\left( \frac{1}{2}mv^2+mgH\right) -0  \\
		      F\left( \frac{H}{\sin\alpha}\right) -\mu mg\cos\alpha\left( \frac{H}{\sin\alpha}\right) & =\frac{1}{2}(2gH(\cosec \alpha-1))+mgH \\
		      F\left( \frac{H}{\sin\alpha}\right)                                                     & =mgH(\cosec\alpha+\mu\cot\alpha)       \\
		      F                                                                                       & =mg(1+\mu\cos\alpha)                   \\
		      \therefore \text{เป็น} \;\frac{mg(1+\mu\cos\alpha)}{mg}                                  & =1+\mu\cos\alpha\; \text{เท่า}
	      \end{align*}
\end{enumerate}

\end{document}