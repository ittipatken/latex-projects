\documentclass[a4paper,12pt]{article}

\usepackage{fontspec}
\newfontfamily{\defaultfont}{CMU Serif}
\newfontfamily{\thaifont}[Scale=MatchLowercase]{TH Sarabun Chula}

\usepackage{polyglossia}
\setdefaultlanguage{thai}
\setotherlanguages{english}

\usepackage[Latin,Thai]{ucharclasses}
\setDefaultTransitions{\defaultfont}{}
\setTransitionTo{Thai}{\thaifont}

\XeTeXlinebreaklocale "th"
\XeTeXlinebreakskip = 0pt plus 0pt

\linespread{1.25}

\usepackage{amsmath,amsthm,amssymb}
\usepackage{mathrsfs}
\usepackage[ISO]{diffcoeff}
\usepackage{siunitx}
\usepackage[margin=1in]{geometry}
\usepackage{graphicx}
\usepackage{hyperref}
\usepackage{caption}
	
\title{Laplace Transform ในฟิสิกส์ฉบับพื้นฐาน}
\author{อิธิพัฒน์ ธนบดีกาญจน์}

\begin{document}

\maketitle

\section{บทนำ}
Laplace Transform ($\mathscr{L}$) เป็นการแปลงโดเมนเวลา ($t$ domain) เป็นโดเมนความถี่ ($s$ domain)
ซึ่งมีความซับซ้อนและมีความเชื่อมโยงกับ Fourier Transform แต่ผู้เขียนต้องการเน้นการใช้ Laplace Transform แก้ปัญหาฟิสิกส์จึงคัดเลือกเฉพาะส่วนที่สำคัญมานำเสนอในการแก้สมการเชิงอนุพันธ์ซึ่งสามารถทำได้โดยแปลงสมการเชิงอนุพันธ์แล้วจัดรูปสมการให้สามารถแปลงผกผันออกมาเป็นผลเฉลยของสมการเชิงอนุพันธ์นั้นได้

\section{อธิบายบทนิยาม}
สมมติสมการการเคลื่อนที่ ที่เราต้องการหา $v(t)$
\begin{equation*}
	m\dot{v}+\beta v=0
\end{equation*}
ในการแก้สมการเชิงอนุพันธ์โดยทั่วไปจะแก้ด้วยการหาปริพันธ์
\begin{equation*}
	\int\dot{v}\,\dl t+\beta\int v\,\dl t=\int 0\,\dl t
\end{equation*}
แต่การทำเช่นนี้จะทำให้เกิดปัญหา คือ จะเกิดพจน์ $\int v\,\dl t$ ขึ้นซึ่งสังเกตได้ว่าเราจะไม่มีทางอินทิเกรตให้เหลือพจน์ $v(t)$ อย่างเดียวได้ จึงต้องใช้วิธีอินทิเกรตแบบสร้างสรรค์ คือ อินทิเกรตให้ทุกพจน์อยู่ในรูปของ $\int \square\,\dl t$ แล้วค่อยแก้สมการต่อ ระลึกว่ามีการอินทิเกรตรูปแบบหนึ่งที่สามารถทำสิ่งที่เราต้องการได้ คือ Integration by parts แต่เราต้องการฟังก์ชันที่มีอนุพันธ์เป็นตัวเดิมมาช่วยในการทำ Integration by parts เราจึงสมมติ $e^{-st}$ ขึ้นมาช่วยในการอินทิเกรตโดยมีเครื่องหมายลบเนื่องจากสูตรของ Integration by parts นั้น มีลบอยู่แต่เรานิยมให้เป็นบวกมากกว่าจึงใส่ลบหน้า $s$ ไปด้วย ดังนั้นจะได้
\begin{equation*}
	\int \dot{v}e^{-st}\,\dl t=ve^{-st}+s\int ve^{-st}\,\dl t
\end{equation*}
ต่อมาเราต้องการให้พจน์ $ve^{-st}$ เป็นค่าคงที่ค่าหนึ่งเพราะจะทำให้เราคำนวณได้ง่ายขึ้นมาก เราจึงเปลี่ยนเป็นปริพันธ์จำกัดเขตโดยมีขอบเขตบนเป็น $\infty$ เพราะจะทำให้พจน์นั้นเป็น $0$ และให้ขอบเขตล่างเป็น $0$ เพราะจะทำให้พจน์นั้นเป็นค่าคงที่
\begin{equation*}
	\int_0^\infty \dot{v}e^{-st}\,\dl t=\left[ve^{-st}\right]_0^\infty+s\int_0^\infty ve^{-st}\,\dl t=-v(0)+s\int_0^\infty ve^{-st}\,\dl t
\end{equation*}
โดย $v(0)$ คือ เงื่อนไขตั้งต้น (Initial condition) ที่โจทย์จะกำหนดมาหรือให้เราทำความเข้าใจสถานการณ์เอง (เช่น ปล่อยวัตถุให้ตกลงสู่พื้นจากหยุดนิ่งจะได้ $v(0)=0$) จากนั้นนิยามฟังก์ชันใหม่ขึ้นมาเพื่อความสะดวกในการเขียนการอินทิเกรตแบบนี้ว่า Laplace Transform
\begin{equation*}
	\int_0^\infty \dot{v}e^{-st}\,\dl t\equiv \mathscr{L}\{\dot{v}\}=-v(0)+s\int_0^\infty ve^{-st}\,\dl t
\end{equation*}
และนิยามอีกฟังก์ชันตามความนิยม คือ
\begin{equation*}
	\int_0^\infty ve^{-st}\,\dl t\equiv \mathscr{L}\{v\}\equiv V(s)
\end{equation*}
ที่นิยามแบบนี้เพราะในที่สุด $V(s)$ จะเป็นฟังก์ชันที่นำไปสู่ผลเฉลยของสมการโดยการทำ Inverse Laplace Transform ($\mathscr{L}^{-1}$) และเราอาจจะเขียน $V(s)$ ย่อ ๆ เป็น $V$ ทำให้สุดท้ายแล้วจะได้
\begin{equation*}
	\mathscr{L}\{\dot{v}\}=-v(0)+sV
\end{equation*}
และเขียนนิยามทั่วไปของ Laplace Transform ได้เป็น
\begin{equation}
	\boxed{
		\mathscr{L}\{f(t)\}=\int_0^\infty f(t)e^{-st}\,\dl t
	}
\end{equation}
เราจะเห็นว่าการที่เราจะทำ Laplace Transform ได้นั้น เราต้องอินทิเกรตทุกพจน์ของสมการ จึงมีผู้ทำตาราง Laplace Transform ไว้เรียบร้อยแล้วให้เรานำมาใช้งานได้โดยสะดวก (พิจารณาเฉพาะ $t\geqslant0$)
\begin{center}
	\bgroup
	\def\arraystretch{2}
	\begin{tabular}{|c|c|c|}
		\hline
		ลำดับที่ & $f(t)$     & $\mathscr{L}\{f(t)\}$ \\
		\hline
		1     & 1          & $\dfrac{1}{s}$        \\
		\hline
		2     & $t$        & $\dfrac{1}{s^2}$      \\
		\hline
		3     & $t^n$      & $\dfrac{n!}{s^{n+1}}$ \\
		\hline
		4     & $e^{-at}$  & $\dfrac{1}{s+a}$      \\
		\hline
		5     & $te^{-at}$ & $\dfrac{1}{(s+a)^2}$  \\
		\hline
		6     & $\sin at$  & $\dfrac{a}{s^2+a^2}$  \\
		\hline
		7     & $\cos at$  & $\dfrac{s}{s^2+a^2}$  \\
		\hline
	\end{tabular}
	\egroup
	\captionof{table}{Laplace Transform บางส่วน ($a$ เป็นค่าคงที่\textenglish{)}}
\end{center}
\begin{center}
	\bgroup
	\def\arraystretch{2}
	\begin{tabular}{|c|c|c|}
		\hline
		ลำดับที่ & $g(t)$               & $\mathscr{L}\{g(t)\}$                                       \\
		\hline
		1     & $\diff*{f(t)}{t}$    & $sF(s)-f(0)$                                                \\
		\hline
		2     & $\diff*[2]{f(t)}{t}$ & $s^2F(s)-sf(0)-f^{(1)}(0)$                                  \\
		\hline
		3     & $\diff*[n]{f(t)}{t}$ & $s^nF(s)-s^{n-1}f(0)-s^{n-2}f^{(1)}(0)-\cdots-f^{(n-1)}(0)$ \\
		\hline
	\end{tabular}
	\egroup
	\captionof{table}{สมบัติของ Laplace Transform บางส่วน}
\end{center}
จากตารางทำให้เราทราบว่าถ้าเราพบสมการเชิงอนุพันธ์อันดับสูงมาก ๆ การใช้ Laplace Transform จะช่วยให้เราแก้ได้ง่ายขึ้นเป็นอย่างมาก
\section{ตัวอย่างโจทย์ปัญหา}
\indent ปล่อยวัตถุมวล $m$ จากหยุดนิ่งที่มีความสูง $h$ วัดจากพื้นในสนามโน้มถ่วง $g$ และมีแรงต้านอากาศ $\beta v$\\
จงหา (ด้วยวิธี Laplace Transform)
\begin{enumerate}
	\item $v(t)$
	\item $x(t)$
\end{enumerate}
จาก
\begin{equation}
	\sum F=ma
\end{equation}
จะได้
\begin{equation}
	mg-\beta v=ma
\end{equation}\label{eq:meq}
ทำ Laplace Transform
\begin{align*}
	\mathscr{L}\{mg-\beta v                   & =ma\}                    \\
	\mathscr{L}\{mg-\beta v                   & =m\dot{v}\}              \\
	mg\mathscr{L}\{1\}-\beta \mathscr{L}\{v\} & =m\mathscr{L}\{\dot{v}\} \\
	\dfrac{mg}{s}-\beta V                     & =m(sV-v(0))
\end{align*}
เนื่องจากปล่อยวัตถุจากหยุดนิ่ง ดังนั้น $v(0)=0$
\begin{align*}
	V & =\dfrac{mg}{s(ms+\beta)}         \\
	V & =\dfrac{g}{s(s+\frac{\beta}{m})}
\end{align*}
แยกเศษส่วนย่อย
\begin{align*}
	V                   & =\dfrac{mg}{\beta s}-\dfrac{mg}{\beta(s+\frac{\beta}{m})} \\
	\mathscr{L}\{v(t)\} & =\dfrac{mg}{\beta s}-\dfrac{mg}{\beta(s+\frac{\beta}{m})}
\end{align*}
ทำ Inverse Laplace Transform
\begin{align*}
	v(t) & =\mathscr{L}^{-1}\left\{\dfrac{mg}{\beta s}-\dfrac{mg}{\beta(s+\frac{\beta}{m})}\right\}                                                 \\
	v(t) & =\dfrac{mg}{\beta}\left(\mathscr{L}^{-1}\left\{\dfrac{1}{s}\right\}-\mathscr{L}^{-1}\left\{\dfrac{1}{(s+\frac{\beta}{m})}\right\}\right)
\end{align*}
จากการเปิดตารางจะได้ผลเฉลยคือ
\begin{equation*}
	v(t)=\dfrac{mg}{\beta}\left(1-e^{-\frac{\beta}{m}t}\right)\;\blacksquare
\end{equation*}
จาก (\ref{eq:meq}) จะได้
\begin{equation*}
	mg-\beta\dot{x}=m\ddot{x}\\
\end{equation*}
ทำ Laplace Transform
\begin{align*}
	\mathscr{L}\{mg-\beta\dot{x}                    & =m\ddot{x}\}              \\
	mg\mathscr{L}\{1\}-\beta \mathscr{L}\{\dot{x}\} & =m\mathscr{L}\{\ddot{x}\} \\
	\dfrac{mg}{s}-\beta(sX-x(0))                    & =m(s^2-sx(0)-\dot{x}(0))
\end{align*}
เนื่องจากปล่อยวัตถุจากความสูง $h$ และหยุดนิ่ง ดังนั้น $x(0)=-h, \dot{x}(0)=0$
\begin{align*}
	\dfrac{mg}{s}-\beta sX-\beta h & =ms^2X+msh                                 \\
	X                              & =\dfrac{mg-s\beta h-mhs^2}{ms^3+\beta s^2}
\end{align*}
แยกเศษส่วนย่อย
\begin{align*}
	X                   & =\dfrac{mg}{s^2(ms+\beta)}-\dfrac{sh(ms+\beta)}{s^2(ms+\beta)}                                      \\
	X                   & =\dfrac{g}{s^2(s+\frac{\beta}{s})}-\dfrac{h}{s}                                                     \\
	X                   & =\dfrac{mg}{\beta s^2}-\dfrac{m^2g}{\beta^2s}+\dfrac{m^2g}{\beta^2(s+\frac{\beta}{m})}-\dfrac{h}{s} \\
	\mathscr{L}\{x(t)\} & =\dfrac{mg}{\beta s^2}-\dfrac{m^2g}{\beta^2s}+\dfrac{m^2g}{\beta^2(s+\frac{\beta}{m})}-\dfrac{h}{s}
\end{align*}
ทำ Inverse Laplace Transform
\begin{align*}
	x(t) & =\mathscr{L}^{-1}\left\{\dfrac{mg}{\beta s^2}-\dfrac{m^2g}{\beta^2s}+\dfrac{m^2g}{\beta^2(s+\frac{\beta}{m})}-\dfrac{h}{s}\right\}                                                                                                                            \\	x(t)&=\mathscr{L}^{-1}\left\{\dfrac{mg}{\beta}\left(\dfrac{1}{s^2}+\dfrac{m}{\beta}\left(\dfrac{1}{s+\frac{\beta}{m}}-\dfrac{1}{s}\right)\right)-\dfrac{h}{s}\right\}\\
	x(t) & =\dfrac{mg}{\beta}\left(\mathscr{L}^{-1}\left\{\dfrac{1}{s^2}\right\}+\dfrac{m}{\beta}\left(\mathscr{L}^{-1}\left\{\dfrac{1}{s+\frac{\beta}{m}}\right\}-\mathscr{L}^{-1}\left\{\dfrac{1}{s}\right\}\right)\right)-\mathscr{L}^{-1}\left\{\dfrac{h}{s}\right\}
\end{align*}
จากการเปิดตารางจะได้ผลเฉลยคือ
\begin{equation*}
	x(t)=\dfrac{mg}{\beta}\left(t+\dfrac{m}{\beta}\left(e^{-\frac{\beta}{m}t}-1\right)\right)-h\;\blacksquare
\end{equation*}
\end{document}