\documentclass[a4paper,12pt]{article}

\usepackage{fontspec}
\newfontfamily{\defaultfont}{CMU Serif}
\newfontfamily{\thaifont}[Scale=MatchLowercase]{TH Sarabun Chula}

\usepackage{polyglossia}
\setdefaultlanguage{thai}
\setotherlanguages{english}

\usepackage[Latin,Thai]{ucharclasses}
\setDefaultTransitions{\defaultfont}{}
\setTransitionTo{Thai}{\thaifont}

\XeTeXlinebreaklocale "th"
\XeTeXlinebreakskip = 0pt plus 0pt

\linespread{1.25}

\usepackage{amsmath,amsthm,amssymb}
\usepackage[ISO]{diffcoeff}
\usepackage{siunitx}
\usepackage[margin=2cm]{geometry}
\usepackage{hyperref}

\renewcommand{\labelenumii}{\theenumii}
\renewcommand{\theenumii}{\theenumi.\arabic{enumii}.}

\begin{document}
\thispagestyle{empty}
\begin{center}
	{\huge \textbf{เฉลยข้อสอบคัดเข้าค่ายฟิสิกส์สอวน. ม.\textenglish{4} 2549}}\\
	พิมพ์โดย Ittipat\\
\end{center}
\begin{enumerate}
	\item \(4 + 2\sqrt{3}\,\si{s} = \SI{7.464}{s} = \SI{7.5}{s}\)
	\item  a = \(\dfrac{\left(v_A - v_B\right)^2}{2d}\)
	\item ชนกันที่ความสูง \(\dfrac{3u^2}{8g}\)
	\item \(m = \SI{25}{g}\)
	\item ดัชนีหักเห \(= \sqrt 2 = 1.414\)
	\item \(v = u \sqrt{\dfrac{\sin \theta - \mu \cos \theta}{\sin \theta + \mu \cos \theta}}\)
	\item \( \theta = \arctan \left (1/(2\mu_s)\right )\)
	\item โวลต์มิเตอร์อ่านค่าได้ \(\SI{40}{V}\)
	\item ความยาวโฟกัสเท่ากับ \(\SI{40}{m}\)
	\item เบนไป \(\SI{40}{\degree}\)
	\item \
	\begin{enumerate}
		\item ความเร่งมีทิศขึ้น ขนาด \(\SI{2.7}{m/s^2}\)
		\item เมื่อที่แขวนหลุด ตาชั่งอ่านได้ศูนย์
	\end{enumerate}
	\item \
	\begin{enumerate}
		\item ขนาดความเร็วต้น \(= \SI{20}{m/s}\)
		\item ช่วงเวลาที่แรงกระทำ \(= \SI{1.0}{s}\)
	\end{enumerate}
	\item \
	\begin{enumerate}
		\item การกระจัดเท่ากับ \(+2\,\si{m}\)
		\item ระยะทางที่อนุภาคเคลื่อนที่ได้เท่ากับ \(\SI{8}{m}\)
	\end{enumerate}
\end{enumerate}
\vfill	
\begin{center}
	\href{http://mpec.sc.mahidol.ac.th/forums/}{mpec.sc.mahidol.ac.th/forums}
\end{center}
\end{document}