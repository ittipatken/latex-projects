\documentclass[a4paper,12pt]{article}

\usepackage{fontspec}
\newfontfamily{\defaultfont}{CMU Serif}
\newfontfamily{\thaifont}[Scale=MatchLowercase]{TH Sarabun Chula}

\usepackage{polyglossia}
\setdefaultlanguage{thai}
\setotherlanguages{english}

\usepackage[Latin,Thai]{ucharclasses}
\setDefaultTransitions{\defaultfont}{}
\setTransitionTo{Thai}{\thaifont}

\XeTeXlinebreaklocale "th"
\XeTeXlinebreakskip = 0pt plus 0pt

\linespread{1.25}

\usepackage{amsmath,amsthm,amssymb}
\usepackage[ISO]{diffcoeff}
\usepackage{siunitx}
\usepackage[margin=2cm]{geometry}
\usepackage{hyperref}

\renewcommand{\labelenumii}{\theenumii}
\renewcommand{\theenumii}{\theenumi.\arabic{enumii}.}

\begin{document}
\thispagestyle{empty}
\begin{center}
	{\huge \textbf{เฉลยข้อสอบคัดเข้าค่ายฟิสิกส์สอวน. ม.\textenglish{4} 2553}}\\
	พิมพ์โดย Ittipat\\
\end{center}
\begin{enumerate}
	\item \
	\begin{enumerate}
		\item \SI{800}{kg/m^3}
		\item \SI{0.1}{m}
	\end{enumerate}
	\item \(\arcsin (0.33)\)
	\item ภาพสูง  \SI{1.14}{cm}  เป็นภาพเสมือน หัวตั้ง
	\item \SI{5.86}{W}, \SI{3.52}{W}
	\item \SI{19}{\celsius}
	\item \(P\propto v^3\)
	\item \
	\begin{enumerate}
		\item \SI{15.8}{m/s} ทิศตะวันตก
		\item \SI{6.3}{m/s^2}
		\item \SI{6.6}{m/s^2}
	\end{enumerate}
	\item มีพลังงานจลน์ \(=mgH(e^{2n})\), กระดอนขึ้นไปสูง \(=He^{2n}\)  
	\item \
	\begin{enumerate}
		\item \(\mu_k mg\)  ทิศไปทางซ้ายมือ
		\item \(\mu_k\left( \dfrac{m}{M}\right)  g \)   ทิศไปทางขวามือ
		\item \(\dfrac{u^2}{2\mu_k g} \left(\dfrac{mM}{(m+M)^2}\right)\)  เทียบกับโลก
		\item \(\dfrac{u^2}{2\mu_k g}\left(1- \dfrac{m}{M}\right)\)
		\item \SI{0}{N}  ไม่มีทิศทาง
	\end{enumerate}
	\item \SI{0}{m/s^2}
\end{enumerate}
\vfill	
\begin{center}
	\href{http://mpec.sc.mahidol.ac.th/forums/}{mpec.sc.mahidol.ac.th/forums}
\end{center}
\end{document}