\documentclass[a4paper,12pt]{article}

\usepackage{fontspec}
\newfontfamily{\defaultfont}{CMU Serif}
\newfontfamily{\thaifont}[Scale=MatchLowercase]{TH Sarabun Chula}

\usepackage{polyglossia}
\setdefaultlanguage{thai}
\setotherlanguages{english}

\usepackage[Latin,Thai]{ucharclasses}
\setDefaultTransitions{\defaultfont}{}
\setTransitionTo{Thai}{\thaifont}

\XeTeXlinebreaklocale "th"
\XeTeXlinebreakskip = 0pt plus 0pt

\linespread{1.25}

\usepackage{amsmath,amsthm,amssymb}
\usepackage[ISO]{diffcoeff}
\usepackage{siunitx}
\usepackage[margin=2cm]{geometry}
\usepackage{hyperref}

\renewcommand{\labelenumii}{\theenumii}
\renewcommand{\theenumii}{\theenumi.\arabic{enumii}.}

\begin{document}
\thispagestyle{empty}
\begin{center}
	{\huge \textbf{เฉลยข้อสอบคัดเข้าค่ายฟิสิกส์สอวน. ม.\textenglish{5} 2551}}\\
	พิมพ์โดย Ittipat\\
\end{center}
\begin{enumerate}
	\item \(\theta_{0}=\frac{1}{2}\arcsin(gD/u^2)\) และ \(90^{\circ}-\frac{1}{2}\arcsin(gD/u^2)\)
	\item ผลบวก \(= 90^{\circ}\)
	\item มุม \(= 30^{\circ}\)
	\item อัตราเร็วของ \(A = 50 \%\) ของอัตราเร็วตั้งต้น
	\item จุด A อยู่สูงจากพื้นระดับ \(= 2.5\) เท่าของรัศมีของราง OB
	\item ความถ่วงจำเพาะของ \(M = 1+d/h\)
	\item แนวแสงออกทำมุม \(= i_1+i_2-A\) องศากับแนวแสงเข้า
	\item อัตราส่วน = \(\dfrac{\mathcal{E}_1R_2-\mathcal{E}_2R_1}{\mathcal{E}_1R_2+\mathcal{E}_2R_1}\)
	\item พลังงานไฟฟ้าที่เก็บอยู่ในตัวเก็บประจุ \(C = \dfrac{1}{2}C\left [ \dfrac{R_2\mathcal{E}_1}{R_1+R_2} \right ]^2\)
	\item ความเร็วมีขนาด \(= \dfrac{2}{5}\dfrac{I^2R}{P_aA+Mg}\)
	\item ขนาดความเร็วของเรือเป็น \(\dfrac{t_2-t_1}{2T+t_2-t_1}\times 100\) เปอร์เซ็นต์ของอัตราเร็วของเสียงในอากาศ
	\item ปริมาตรน้ำที่เหลือในถัง \(= \pi R^2 (H-R\tan \theta)\)
	\item จุดที่ P ชนขั้นบันไดครั้งแรกอยู่ห่างจากกำแพง\( = \SI{0.547}{m} + \SI{0.20}{m}= \SI{0.747}{m}\)
\end{enumerate}
\vfill	
\begin{center}
	\href{http://mpec.sc.mahidol.ac.th/forums/}{mpec.sc.mahidol.ac.th/forums}
\end{center}
\end{document}