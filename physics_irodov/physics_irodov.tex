\documentclass[a4paper,12pt]{article}

\usepackage{fontspec}

\setmainfont{CMU Serif}
\setsansfont{TeX Gyre Heros}
\setmonofont{TeX Gyre Cursor}

\newfontfamily{\thaifont}[Scale=MatchUppercase]{TH Sarabun Chula}
\newfontfamily{\englishfont}{CMU Serif}

\usepackage[Latin,Thai]{ucharclasses}
\setTransitionTo{Thai}{\thaifont}
\setTransitionFrom{Thai}{\thaifont}

\usepackage{setspace}
\onehalfspacing

\usepackage{amsmath,amsthm,amssymb}

\begin{document}
	
	\section*{สรุปความรู้จลนศาสตร์}
เวกเตอร์เฉลี่ยของความเร็วและความเร่งของอนุภาค
~\\ $\bullet <\vec{v}> = \dfrac{\Delta \vec r}{\Delta t},\; <\vec{v}>= \dfrac{\Delta \vec v}{\Delta t}$~\\
$\Delta \vec r$ คือเวกเตอร์การกระจัด

ความเร็วและความเร่งของอนุภาค
~\\ $\bullet \vec v = \dfrac{d\vec r}{dt}, \; \vec a = \dfrac{d\vec v}{dt}$

ความเร่งของอนุภาคที่จุดๆหนึ่งในแนวสัมผัสและตั้งฉากกับเส้นวิถีการเคลื่อนที่ ~\\ $\bullet a_\tau = \dfrac{dv_\tau}{dt}, \; a_n = \dfrac{v^2}{R}$
$R$ คือรัศมีความโค้งของเส้นวิถีที่จุดนั้น


ระยะทางที่อนุภาคเคลื่อนที่ได้ ~\\ $\bullet s = \int v dt$ 
$v$ คือขนาดของเวกเตอร์ความเร็วของอนุภาค

 ความเร็วเชิงมุมและความเร่งเชิงมุมของวัตถุแข็งเกร็ง~\\ 
$\bullet \vec \omega = \dfrac{d\vec \phi}{dt}, \; \vec \alpha = \dfrac{d\vec \omega}{dt}$

ความสัมพันธ์ระหว่างปริมาณเชิงเส้นและปริมาณเชิงมุมสำหรับวัตถุแข็งเกร็งที่กำลังหมุน~\\ 
$\bullet \vec v = \vec \omega \times \vec r, \;\;\; a_n = \omega ^2 R, \; |a_\tau| = \alpha R$\\
\indent $\vec r$ คือเวกเตอร์ตำแหน่งของจุดที่กำลังพิจารณาเทียบกับจุดใดๆบนแกนหมุนและ\\
\indent $R$ คือระยะห่างของจุดจากแกนหมุน

\pagebreak

	\section*{สรุปความรู้เรื่องอุณหพลศาสตร์}
2.1 สรุปความรู้เรื่องสมการสถานะของก๊าซ

~\\
$\bullet$  กฎของก๊าซอุดมคติ
$\displaystyle PV=\frac{m}{M}RT$
โดยที่ M คือมวลโมลาร์

~\\ $\bullet$  สมการบาโรเมทริก
$\displaystyle P=P_0e^{\frac{-Mgh}{RT}}$
เมื่อ $P_0$คือความดันที่ระดับ $h=0$

~\\ $\bullet$ สมการสถานะของก๊าซแวน เดอร์ วาลส์
$\displaystyle(P_{van}+\frac{a^2}{V_M^2})(V_M-b)=nRT$
เมื่อ $V_M$คือปริมาตรของก๊าซ $1 \mbox{ mol}$ ซึ่งขึ้นกับความดันและอุณหภูมิ

\noindent 2.2 สรุปความรู้เรื่อง กฏข้อที่หนึ่งของเทอร์โมไดนามิกส์ และค่าความจุความร้อน

~\\ $\bullet$  กฏข้อที่ 1 ของเทอร์โมไดนามิกส์
$Q=\Delta U+W$
เมื่อ $\Delta$ Uเป็นพลังงานภายในที่เพิ่มขึ้นของระบบ

~\\ $\bullet$ งานที่ทำโดยระบบ
$W=\int pdV$

~\\ $\bullet$  พลังงานภายในของระบบ
$U=\displaystyle{\frac{m}{M}}C_VT=\displaystyle{\frac{m}{M}}\displaystyle{\frac{RT}{\gamma -1}}=\displaystyle{\frac{pV}{\gamma -1}}$

ค่าความจุความร้อนต่อโมล ในกระบวนการ โพลีโทรปิก$(pV^n=const)$
$C=\displaystyle{{\gamma -1}}-\displaystyle{\frac{T}{n-1}}=\displaystyle{\frac{(n-\gamma)R}{(n-1)(\gamma -1)}}$

~\\ $\bullet$ พลังงานภายใน ของก๊าซแวน เดอ วาลส์ 1โมล
$U=C_VT-\displaystyle{\frac{a}{V_M}}$
โดยที่ $V_M$เป็นปริมาตรของก๊าซ $1$ โมล

2.3 สรุปความรู้เรื่องทฤษฎีจลย์ของก๊าซ,กฎของโบลซ์แมนและ Maxwell's distribution

~\\ $\bullet$ สมการสภาวะของก๊าซอุดมคติ (Equation of an ideal gas state)
p=nkT

~\\ $\bullet$ พลังงานเฉลี่ยของโมเลกุล
$<e>=\displaystyle{\frac{i}{2}}kT$
โดยที่ iเป็นผลรวม degree of freedom ของการเลื่อนที่ การหมุน และการสั่น

~\\ $\bullet$ Maxwellian distribution
$dN(v_x)=N(\displaystyle {\frac{m}{2\pi kT}})^{\frac{1}{2}}e^{\frac{-m{v_x}^2}{2kT}}d{v_x}
dN(v)=N(\displaystyle {\frac{m}{2\pi kT}})e^{\frac{-m{v}^2}{2kT}}d{v}$

~\\ $\bullet$ ความเร็วที่มีโอกาสพบอนุภาคมากที่สุด,ความเร็วเฉลี่ย,รากที่สองของความเร็วกำลังสองเฉลี่ย
$v_p=\sqrt{\displaystyle {\frac{2kT}{m}}}; <v>=\sqrt{\frac{8kT}{\pi m}}; v_{sq}=\sqrt{\frac{3kT}{m}}$

~\\ $\bullet$ สูตรของ โบลซ์แมน
$n=n_0e^{\frac{U-U_0}{kT}}$
โดยที่ Uเป็นr]พลังงานศักย์ของโมเลกุล

2.4 สรุปความรู้เรื่องกฎข้อที่ 2 ของเทอร์โมไดนามิกส์ และเอนโทรปี

~\\ $\bullet$ ประสิทธิภาพของวัฏจักรความร้อน
$\eta=\displaystyle{\frac{W}{Q_1}}=1-\displaystyle{\frac{Q_2}{Q_1}}$
โดยที่ $Q_1$เป็นความร้อน ที่วัฎจักรได้รับจากแหล่งที่มีอุณหภูมิสูงและ $Q_2$เป็นความร้อนที่วัฎจักรปล่อยไปให้แหล่งที่มีอุณหภูมิต่ำกว่า

~\\ $\bullet$ ประสิทธิภาพของวัฏจักรคาโนต์
$\eta=1-\displaystyle{\frac{T_2}{T_1}}$
โดยที่ $T_1$เป็นอุณหภูมิ ของแหล่งที่มีอุณหภูมิสูงและ $T_2$เป็นอุณหภูมิของแหล่งที่มีอุณหภูมิต่ำกว่า

~\\ $\bullet$ บทบัญญัติของคลอซิอุส
$0>\int \displaystyle{\frac{\delta Q}{T}}$

~\\ $\bullet$ เอนโทรปีที่เพิ่มขึ้นในระบบ
$\Delta S>\int\displaystyle{\frac{\delta Q}T{}}$

~\\ $\bullet$ ความสัมพันธ์พื้นฐานของเทอร์โมไดนามิกส์
$TdS>dU+pdV$

~\\ $\bullet$ ควาสัมพันธ์ระหว่าง เอนโทรปี กับ statistical weight (the thermodynamics)
$S=k\ln\Omega$

\pagebreak

	\section*{สรุปความรู้เรื่องพลศาสตร์ไฟฟ้า}
3.1 สรุปความรู้เรื่องสนามไฟฟ้าในสุญญากาศที่ไม่แปรตามเวลา

~\\ $\bullet$    ความเข้มและศักย์ของสนามไฟฟ้าจากประจุจุด q :
$\vec{E} = \dfrac{1}{4\pi\epsilon_{0}}\dfrac{q}{r^{3}}\vec{r}       \varphi = \dfrac{1}{4\pi\epsilon_{0}}\dfrac{q}{r}               (3.1a)$

~\\ $\bullet$    ความสัมพันธ์ระหว่างความเข้มสนามไฟฟ้ากับศักย์ไฟฟ้า :
$\vec{E} = -\vec{\nabla}\varphi                   (3.1b)$

~\\ $\bullet$    ทฤษฎีบทของเกาส์ และการหมุนวนของเวกเตอร์ $\vec{E}$ :
$\displaystyle \oint \vec{E}\cdot d\vec{A} = \dfrac{q}{\epsilon_{0}}            \displaystyle \oint \vec{E}\cdot d\vec{r} = 0                          (3.1c)$

~\\ $\bullet$    ศักย์ไฟฟ้าและความเข้มสนามไฟฟ้าของขั้วคู่ไฟฟ้าของประจุจุดซึ่งมีโมเมนต์ไฟฟ้า $\vec{p}$ :
$\varphi = \dfrac{1}{4\pi\epsilon_{0}}\dfrac{\vec{p}\cdot\vec{r}}{r^{3}}            E= \dfrac{1}{4\pi\epsilon_{0}}\dfrac{p}{r^{3}}\sqrt{1+3\cos^{2}\theta}                  (3.1d)$
โดยที่ $\theta$ คือมุมระหว่างเวกเตอร์ $\vec{r}$ กับ $\vec{p}$

~\\ $\bullet$    พลังงาน  W  ของขั้วคู่ไฟฟ้า  $\vec{p}$ ในสนามไฟฟ้าภายนอก และทอร์ก $\vec{T}$ กระทำต่อขั้วคู่ไฟฟ้า :
$W=-\vec{p}\cdot\vec{E}                  \vec{T} = \vec{p}\times\vec{E}$              (3.1e)

~\\ $\bullet$    แรง $\vec F$ ที่กระทำต่อขั้วคู่ไฟฟ้า และเงาฉายของมัน $F_{x} :
\vec{F} = p\dfrac{\partial \vec{E}}{\partial l}                F_{x} = \vec{p}\cdot\vec{\nabla}E_{x}$                       (3.1f)
โดยที่ $\dfrac{\partial \vec{E}}{\partial l}$ เป็นอนุพันธ์ของเวกเตอร์ $\vec{E}$ เทียบกับทิศทางของขั้วคู่ไฟฟ้า   $\vec{\nabla}E_{x}$ คือเกรเดียนท์ของฟังก์ชัน $E_{x}$~\\
3.2 ตัวนำไฟฟ้าและไดอิเล็กทริกในสนามไฟฟ้า

~\\ $\bullet$ สนามไฟฟ้าใกล้ผิวของตัวนำในสุญญากาศ:
$E_{n} = \dfrac{\sigma}{\epsilon_{0}}$           (3.2a)

~\\ $\bullet$ ฟลักซ์ของเวคเตอร์การโพลาร์ไรซ์ $\vec{P}$ ผ่านผิวปิด :
$\displaystyle \oint \vec{P}\cdot d\vec{A} = - q^\prime$          (3.2b)
โดยที่ $q^\prime$  คือผลบวกเชิงพีชคณิตของประจุไม่อิสระที่ถูกล้อมโดยผิวนี้

~\\ $\bullet$ เวคเตอร์ $\vec{D}$ และกฎของเกาส์สำหรับมัน :
$\vec{D} = \epsilon_{0}\vec{E} + \vec{P}              \displaystyle \oint\vec{D}\cdot \vec{dA} = q$          (3.2c)
โดย q คือผลบวกเชิงพีชคณิตของประจุอิสระภายในผิวปิดนี้

~\\ $\bullet$ ความสัมพันธ์ระหว่างไดอิเล็กทริกสองชนิดที่ผิวขอบ:
$P_{2n} - P{1n} = -\sigma^\prime   ,      D_{2n} - D_{1n} = \sigma   ,      E_{2\tau} = E_{1\tau}$            (3.2d)
โดย $\sigma$  และ $\sigma^\prime$  เป็นความหนาแน่นประจุเชิงพื้นที่ของประจุอิสระ และไม่อิสระตามลำดับ และเวคเตอร์หน่วย $\hat{n}$ ชี้ตั้งฉากจากตัวกลาง 1 ไปตัวกลาง 2

~\\ $\bullet$ ในไดอิเล็กทริกไอโซทรอปิก:
$\vec{P} = \chi\epsilon_{0}\vec{E} ,   \vec{D} = \epsilon\epsilon_{0}\vec{E} ,   \epsilon = 1+\chi$            (3.2e)

~\\ $\bullet$ ในกรณีไดอิเล็กทริกไอโซทรอปิกที่สม่ำเสมอถูกเติมเต็มเข้าในช่องว่างระหว่างผิวสมศักย์ :
$\vec{E} = \dfrac{\vec{E}_{0}}{\epsilon_{0}}$~\\

3.3 ความจุไฟฟ้า และพลังงานของสนามไฟฟ้า

~\\ $\bullet$ ความจุของตัวเก็บประจุแบบแผ่นขนาน :
$C = \epsilon\epsilon_{0}\dfrac{A}{d}$                         (3.3a)

~\\ $\bullet$ พลังงานที่กระทำระหว่างกันของประจุจุด :
$W = \dfrac{1}{2}\displaystyle \sum q_{i}\varphi_{i}$                  (3.3b)

~\\ $\bullet$ พลังงานไฟฟ้าทั้งหมดของระบบซึ่งประจุกระจายอย่างต่อเนื่อง :
$W = \dfrac{1}{2}\displaystyle \int \varphi\rho dV$                     (3.3c)

~\\ $\bullet$ พลังงานไฟฟ้ารวมของวัตถุมีประจุสองตัว 1 และ 2 :
$W = W_{1}+W_{2}+W_{12}$                            (3.3d)
เมื่อ $W_{1}$ และ $W_{2}$ คือพลังงานในตัวเองของวัตถุ และ $W_{12}$ คือพลังงานระหว่างกันของวัตถุ 2 ก้อนนั้น

~\\ $\bullet$ พลังงานของตัวเก็บประจุที่มีประจุอยู่ :
$W = \dfrac{qV}{2} = \dfrac{q^{2}}{2C} = \dfrac{CV^{2}}{2}$                    (3.3e)

~\\ $\bullet$ ความหนาแน่นเชิงปริมาตรของพลังงานสนามไฟฟ้า :
$\omega = \dfrac{\vec{E}\cdot\vec{D}}{2} = \dfrac{\epsilon\epsilon_{0}E^{2}}{2}$                    (3.3f)~\\

3.4 กระแสไฟฟ้า

~\\ $\bullet$ กฎของโอห์มสำหรับส่วนของวงจรแบบไม่เอกพันธ์ :
$I = \dfrac{V_{12}}{R} = \dfrac{\varphi_{1} - \varphi_{2} +\varepsilon_{12}}{R}$                (3.4a)
โดยที่ $V_{12}$ คือความต่างศักย์ตกคร่อมส่วนย่อยนั้น

~\\ $\bullet$ รูปแบบดิฟเฟอเรนเชียลของกฎของโอห์ม :
$\vec{j} = \sigma(\vec{E}+\vec{E}^{*})$                               (3.4b)
โดยที่ $\vec{E}^{*}$ คือความเข้มสนามจากแรงภายนอก

~\\ $\bullet$ กฎของ Kirchhoff (สำหรับวงจรไฟฟ้า) :
$\displaystyle \sum I_{k} = 0,       \displaystyle \sum I_{k}R_{k} = \displaystyle \sum \varepsilon_{k}$                               (3.4c)

~\\ $\bullet$ กำลัง $P$ ของกระแส และกำลังไฟฟ้าที่เปลี่ยนเป็นความร้อน $Q$ :
$P = VI = (\varphi_{1}-\varphi_{2}+\varepsilon_{12})I  ,          Q = RI^{2}$                       (3.4d)

~\\ $\bullet$ กำลังไฟฟ้าจำเพาะ(specific power) $P_{sp}$ และ $Q_{sp}$ ของกระแสไฟฟ้า :
$P_{sp} = \vec{j}\cdot(\vec{E}+\vec{E}^{*})  ,          Q_{sp} = \rho j^{2}$                   (3.4e)

~\\ $\bullet$ ความหนาแน่นกระแสในโลหะ :
$\vec{j} = en\vec{u}$                      (3.4f)
โดยที่ $\vec{u}$ คือความเร็วเฉลี่ยของอนุภาคที่นำพากระแส

3.5 สรุปความรู้เรื่องสนามแม่เหล็กที่ไม่แปรตามเวลา และแม่เหล็ก

~\\ $\bullet$ สนามแม่เหล็กจากประจุจุดที่เคลื่อนที่ด้วยความเร็วไม่สัมพัทธภาพ  $\vec{v}$  :
$\vec{B} = \dfrac{\mu_{0}}{4\pi}\dfrac{q\vec{v}\times\vec{r}}{r^{3}}                          (3.5a)$

~\\ $\bullet$ กฎของบิโอต์-ซาวาร์ต์ :
$\vec{dB} = \dfrac{\mu_{0}}{4\pi}\dfrac{\vec{j}\times\vec{r}}{r^{3}}dV  ,         \vec{dB} = \dfrac{\mu_{0}}{4\pi}\dfrac{I\vec{dl}\times\vec{r}}{r^{3}}                          (3.5b)$

~\\ $\bullet$ การหมุนวนของ $\vec{B}$ และกฎของเกาส์สำหรับสนามแม่เหล็ก :
$\displaystyle \oint \vec{B}\cdot\vec{dr} = \mu_{0}I ,            \displaystyle \oint \vec{B}\cdot\vec{dA} = 0$                               (3.5c)

~\\ $\bullet$ แรงลอเรนตซ์ :
$\vec{F} = q\vec{E} + q\vec{v}\times\vec{B}$                                       (3.5d)

~\\ $\bullet$ แรงแอมแปร์ :
$\vec{dF} = \vec{j}\times\vec{B} dV ,                 \vec{dF} = I\vec{dl}\times\vec{B}$                       (3.5e)

~\\ $\bullet$ แรงและโมเมนต์ของแรงที่กระทำต่อขั้วคู่แม่เหล็ก $\vec{p}_{m} = IA\hat{n} :
\vec{F}= p_{m}\dfrac{\partial \vec{B}}{\partial n} ,        \vec{T} = \vec{p}_{m}\times\vec{B}$               (3.5f)
โดย $\dfrac{\partial \vec{B}}{\partial n}$ คืออนุพันธ์ของ $\vec{B}$ เทียบกับทิศทางของขั้วคู่แม่เหล็ก

~\\ $\bullet$ การหมุนวนของเวคเตอร์การทำให้เป็นแม่เหล็ก(Magnetization vector)$\vec{J} :
\displaystyle \oint \vec{J}\cdot\vec{dr} = I^\prime                                    (3.5g)$
โดย  $I^\prime$ คือกระแสรวมระดับจุลภาค

~\\ $\bullet$ เวคเตอร์ $\vec{H}$ และการหมุนวนของมัน :
$\vec{H} = \dfrac{\vec{B}}{\mu_{0}} - \vec{J} ,          \displaystyle \oint \vec{H}\cdot\vec{dr} = I                                  (3.5h)$
ดดยที่ I คือผลรวมเชิงพีชคณิตของกระแสระดับมหภาค

~\\ $\bullet$ ความสัมพันธ์ที่ขอบนะหว่างแม่เหล็ก 2 ตัว :
$B_{1n} = B_{2n} ,                H_{1\tau} = H_{2\tau}                                             (3.5i)$

~\\ $\bullet$ กรณีของแม่เหล็กที่มี $\vec{J} = \chi_{m}\vec{H}$ :
$\vec{B} = \mu\mu_{0}\vec{H} ,          \mu = 1+\chi_{m}                                             (3.5j)$~\\

3.6 การเหนี่ยวนำแม่เหล็กไฟฟ้า และสมการของแมกซ์เวลล์

~\\ $\bullet$ กฎการเหนี่ยวนำแม่เหล็ฏไฟฟ้าของฟาราเดย์ :
$\Phi_{m} = N\Phi_{1m}                                                  (3.6b)$
โดย $N$ คือจำนวนขด  $\Phi_{1m}$  คือฟลักซ์แม่เหล็กที่ผ่านแต่ละขด 

~\\ $\bullet$ ค่าความเหนี่ยวนำของโซลีนอยด์ :
$L = \mu\mu_{0}N^{2}\dfrac{A}{l}                                   (3.6c)$

~\\ $\bullet$ พลังงานภายในตัวเหนี่ยวนำ และพลังงานระหว่างกันของตัวเหนี่ยวนำ 2 ตัว :
$W = \dfrac{LI^{2}}{2} ,                 W_{12} = L_{12}I_{1}I_{2}                                (3.6d)$

~\\ $\bullet$ ความหนาแน่นเชิงปริมาตรของพลังงานสนามแม่เหล็ก :
$\omega_{m} = \dfrac{B^{2}}{2\mu\mu_{0}} = \dfrac{\vec{B}\cdot\vec{H}}{2}                     (3.6e)$

~\\ $\bullet$ ความหนาแน่นของกระแสกระจัด  (Displacement current density) :
$\vec{j}_{dis} = \dfrac{\partial \vec{D}}{\partial t}                                                      (3.6f)$

~\\ $\bullet$ สมการของแมกซ์เวลล์ในรูปแบบดิฟเฟอเรนเชียล :
$\vec{\nabla}\times\vec{E} = -\dfrac{\partial \vec{B}}{\partial t}
\vec{\nabla}\cdot\vec{B} = 0
\vec{\nabla}\times\vec{H} = \vec{j} + \dfrac{\partial \vec{D}}{\partial t}
\vec{\nabla}\cdot\vec{E} = \rho                                                                                   (3.6g)$
โดยที่ $\vec{\nabla}\times$  คือเคิร์ล   และ $\vec{\nabla}\cdot$ คือไดเวอร์เจนซ์

~\\ $\bullet$ สูตรการแปลงสนามจากกรอบอ้างอิง $\Sigma$  ไปเป็นกรอบอ้างอิง $\Sigma^\prime$ ซึ่งเคลื่อนที่ด้วยความเร็ว $\vec{v}_{0}$ เมื่อเทียบกับอับแรก :
ในกรณีที่ $v_{0} \ll c$
$\vec{E}^\prime = \vec{E} + \vec{v}_{0}\times\vec{B} ,              \vec{B}^\prime = \vec{B} - \dfrac{\vec{v}_{0}\times\vec{E}}{c^{2}}                                             (3.6h)$
ในกรณีทั่วไป
$\vec{E}^\prime_{\parallel } = \vec{E}_{\parallel} ,             \vec{B}^\prime_{\parallel} = \vec{B}_{\parallel}
\vec{E}^\prime_{\perp} = \dfrac{\vec{E}_{\perp} + \vec{v}_{0}\times\vec{B}}{\sqrt{1-(\dfrac{v}{c})^{2}}} ,                      \vec{B}^\prime_{\perp} = \dfrac{\vec{B}_{\perp}-\dfrac{\vec{v}_{0}\times\vec{E}}{c^{2}}}{\sqrt{1-(\dfrac{v}{c})^{2}}}                                                                     (3.6i)$
โดยที่เครื่องหมาย  $\parallel$  และ  $\perp$  แสดงถึงองค์ประกอบของสนามที่ขนานและตั้งฉากกับทิศทางการเคลื่อนที่ของกรอบอ้างอิง คือเวคเตอร์ $\vec{v}_{0}$ ตามลำดับ

3.7 การเคลื่อนที่ของอนุภาคมีประจุในสนามแม่เหล็กและสนามไฟฟ้า
~\\ $\bullet$แรงโลเร็นตซ์ :
$\vec{F} = q\vec{E}+q\vec{v}\times\vec{B}$                              (3.7a)

~\\ $\bullet$การเคลื่อนที่เชิงสัมพัทธภาพของอนุภาค :
$\dfrac{d}{dt}\dfrac{m_{0}\vec{v}}{\sqrt{1-(\dfrac{v}{c})^{2}}} = \vec{F}$                            (3.7b)

~\\ $\bullet$คาบของการวนครบรอบของประจุในสนามแม่เหล็กสม่ำเสมอ :
$T = \dfrac{2\pi m}{qB}                                                 (3.7c)$
โดยที่ $m$ คือมวลสัมพัทธภาพของอนุภาค, $m = \dfrac{m_{0}}{\sqrt{1 - (\dfrac{v}{c})^{2}}}$

~\\ $\bullet$เงื่อนไขเบตาตรอน คือเงื่อนไขสำหรับอิเล็กตรอนที่เคลื่อนที่เป็นวงกลมภายในเครื่องเร่งเบตาตรอน :
$B_{0} = \dfrac{1}{2}\left\langle B \right\rangle                                (3.7d)$
โดยที่ $B_{0}$ คือสนามแม่เหล็กที่จุดของวงโคจร และ $\left\langle B \right\rangle$  คือค่าเฉลี่ยของสนามแม่เหล็กภายในวงโคจร

\end{document}